\documentclass[12pt,a4paper,UTF8]{ctexart}
\usepackage{geometry}
	\geometry{left=2.5cm,right=2.5cm,top=0cm,bottom=1cm}
\usepackage{xeCJK,amsmath,paralist,enumitem,booktabs,multirow,graphicx,subfig,setspace}
	\setlength{\parindent}{2em}
\usepackage[colorlinks,linkcolor=blue,urlcolor=blue]{hyperref}

%%%%%%%%%%%%%%%%%%%%%%%%%正文开始%%%%%%%%%%%%%%%%%%%%%%%%%%
\begin{document}

\title{\Large\bfseries Measure the atomic emission spectra basing on OMA\footnotemark[1]}
\author{\large\textit{Ziwei Huang}$^{1}$\footnotemark[2] \\ 
\small{1 School of Physics, Sun Yat-sen University, Guangzhou  { \rm 510275}, China}}
\date{}
\maketitle\thispagestyle{empty} 

\begin{spacing}{1.7}
{\bfseries Abstract:}
According to the Bohr's atom model, when an atom jumps from one fixed state to another, it will radiate or absorb photons of a certain frequency, and the energy of the photon is determined by the energy difference between these two fixed states. This discovery successfully explained the discontinuity of the hydrogen atom spectrum and laid a solid foundation for the development of quantum theory. What's more, it also has important applications in the field of component and structure analysis of the matter. The precise measurement of the atomic emission spectra is the physical basis for understanding the atomic leap processes. In this experiment, we use an optical multichannel analyzer (OMA)  based on the reflection grating to measure the atomic emission spectra.

We scanned the atomic emission spectra of nine commonly used laboratory light sources (mercury lamp, sodium lamp, hydrogen-deuterium lamp, bromine tungsten lamp and LED bulbs of five colors) and found that the spectra of mercury, sodium and hydrogen-deuterium lamps are discrete spectra, while the spectra of bromine tungsten lamp and LED bulbs of five colors are continuous spectra. Next, for light sources with discrete spectra, we obtained the calibration relationship by comparing the measured spectrum with the standard spectrum of the mercury lamp , and calibrated the hydrogen-deuterium spectrum accordingly. We also investigated the characteristics of the sodium doublet basing on the spectrum measured at a smaller grating distance. Finally, for light sources with continuous spectra, we used Gaussian fitting method to calculate the central wavelengths and the half-height widths of the wave packets presented in the spectra of LED bulbs, as well as investigated the spectral characteristics of the bromine tungsten lamp.
\par
\bfseries{Key words}: Optical multichannel analyzer (OMA), Atomic emission spectra
\vspace{2em}
\end{spacing}

\renewcommand{\thefootnote}{\fnsymbol{footnote}}
\footnotetext[1]{{Supported and taught by Luyoutang, School of Physics, Sun Yat-sen University}}
\footnotetext[2]{{Corresponding author. E-mail:\ \url{huangzw29@mail2.sysu.edu.cn}}}

\end{document}