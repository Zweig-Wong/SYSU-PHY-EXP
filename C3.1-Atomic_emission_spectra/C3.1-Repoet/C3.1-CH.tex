\documentclass[12pt,a4paper,UTF8]{ctexart}
\usepackage{geometry}
	\geometry{left=2.5cm,right=2.5cm,top=0cm,bottom=2cm}
\usepackage{xeCJK,amsmath,paralist,enumitem,booktabs,multirow,graphicx,subfig,setspace}
	\setlength{\parindent}{2em}
\usepackage[colorlinks,linkcolor=blue,urlcolor=blue]{hyperref}

%%%%%%%%%%%%%%%%%%%%%%%%%正文开始%%%%%%%%%%%%%%%%%%%%%%%%%%
\begin{document}

\title{\LARGE\bfseries 基于OMA的原子发射光谱测量\footnotemark[1]}
\author{\large\textit{黄子维}$^{1}$\footnotemark[2] \\ 
\small{1 \textit{中山大学 物理学院,广东 广州 }510275}}
\date{}
\maketitle
\setcounter{page}{0}
\thispagestyle{empty}

\begin{spacing}{2.0}
{\bfseries 摘 {} 要:}
玻尔原子模型指出,当原子从一种定态跃迁到另一种定态时,将辐射或吸收一定频率的光子,光子的能量由这两个定态的能量差决定。这一发现成功解释了氢原子光谱的不连续性,为量子理论的发展奠定了基础,同时该发现在物质组分和结构分析领域也有着重要应用。对原子发射光谱的精密测量是理解原子跃迁过程的物理基础,本实验中,我们采用基于反射光栅的光学多道分析仪(OMA)测量原子发射光谱。

我们扫描了九种实验室常用光源(汞灯,钠灯,氢氘灯,溴钨灯和五种颜色的LED灯)的原子发射光谱,发现其中汞灯,钠灯和氢氘灯光谱为分立谱,而溴钨灯和五种颜色LED灯光谱为连续谱。接下来,对于分立谱光源,我们使用测量得到的汞灯谱线与标准谱线对比得到标定关系,并据此标定了氢氘灯的谱线,同时我们基于在更小光栅距离下测量得到的钠灯光谱,研究了钠双黄线的特征。最后,对于连续谱光源,我们使用高斯拟合方法计算LED灯的波包中心波长及其半高宽,并研究了溴钨灯光源的光谱特点。
\par
\bfseries{关键词}: 光学多道分析仪(OMA),原子发射光谱
\vspace{2em}
\end{spacing}

\renewcommand{\thefootnote}{\fnsymbol{footnote}}
\footnotetext[1]{由中山大学物理学院陆佑堂提供器材和指导。}
\footnotetext[2]{通信作者,\url{huangzw29@mail2.sysu.edu.cn}}


%%%%%%%%附录:数据处理%%%%%%%
\newpage
\pagestyle{plain}
\hspace{2em}
\begin{center}
\LARGE\textbf{实验C3.1 原子发射光谱}
\end{center}

%%信息
\begin{doublespacing}
	\centering
	\begin{tabular}{ll}
	 & \\
	{\CJKfontspec{Droid Sans Fallback} 实验人:黄子维 20980066} & {\CJKfontspec{Droid Sans Fallback}合作者:黄睿杰 20980062}\\
	{\CJKfontspec{Droid Sans Fallback} 实验时间:2021.9.23~星期四~上午} & {\CJKfontspec{Droid Sans Fallback} 室温:29$^{\circ}$C~相对湿度:62\%}
	\end{tabular}
\end{doublespacing}

\subsection*{【数据处理及分析】}

\subsection*{【思考题】}

\end{document}

